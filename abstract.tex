\newpage
\pagenumbering{roman}
\addcontentsline{toc}{chapter}{\numberline{}Abstract}
\centerline{ABSTRACT}
\begin{center}
{\LARGE Symmetry and Reconstruction \\of Particle Structure from \\ Random Angle Diffraction Patterns }\\[1cm]
by\\[1cm]
Sandi Wibowo \\[2cm]
The University of Wisconsin-Milwaukee, 2016\\
Under the Supervision of Professor Dilano Kerzaman Saldin\\[2.5cm]
\end{center}
The problem of determining the structure of a biomolecule, when all the evidence from experiment consists of individual diffraction patterns from random particle orientations, is the central theoretical problem with an XFEL. 
One of the methods proposed is a calculation over all measured diffraction patterns of the average angular correlations between pairs of points on the diffraction patterns. 
It is possible to construct from these a matrix B characterized by angular momentum quantum number l, and whose elements are characterized by radii q and q' of the resolution shells. 
If matrix B is considered as dot product of vectors, which magnetic quantum number m is the component, singular value of B reveals the number of magnetic quantum numbers in the spherical harmonics expansion. 
What is shown in this paper is dependency of magnetic quantum number on symmetry can be associated to lowest independent parameter to describe symmetry. At the very least this determines information about particle symmetry from experiment data, independent of any assumed symmetry. 
An equally important point is that matrix B provides a means of reconstructing diffraction volume. This can be done by formulating intensity and matrix B as linear equation. Lastly, positivity constraint and optimization method is used to construct diffraction volume and phase is determined from phasing algorithm.
